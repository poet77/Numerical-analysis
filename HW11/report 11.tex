\documentclass{article}
\textheight 23.5cm \textwidth 15.8cm
%\leftskip -1cm
\topmargin -1.5cm \oddsidemargin 0.3cm \evensidemargin -0.3cm
%\documentclass[final]{siamltex}

\usepackage{verbatim}
\usepackage{fancyhdr}
\usepackage{amssymb,ctex}
\usepackage{mathrsfs}
\usepackage{latexsym,amsmath,amssymb,amsfonts,epsfig,graphicx,cite,psfrag}
\usepackage{eepic,color,colordvi,amscd}
\usepackage{enumerate}
\usepackage{booktabs}
\usepackage{graphicx}
\usepackage{float}
\usepackage{multirow}


\title{Numerical Analysis Homework11}
\author{Zhang Jiyao,PB20000204}

\begin{document}
	\maketitle
	
	\section{Introduction}
	
	编写一个用有限差分法求解线性两点边值问题的通用的计算机程序,允许用户提供$\alpha,\beta,a,b,n$,以及函数$u,v,w$.
	
	对下面的例子测试上题中编写的程序:
	$$	
	a.\left\{ 
	\begin{array}{lc}
		x^{''} = -x \\
		x(0) = 0  \quad x(\pi/2)=7\\
	\end{array}
	\right.$$
	
	$$	
	b.\left\{ 
	\begin{array}{lc}
		x^{''} = 2e^t-x \\
		x(0) = 2  \quad x(1)=e+cos1\\
	\end{array}
	\right.$$
	
 然后计算这两种测试情况中数值解的误差,对应的解分别是:$a.x(t)=\text{7}sint+\text{3}cost.$\quad $b.x(t)=e^t+cost$.
	
	
	\section{Method}
	假定我们求解的问题是:
	$$	
	\left\{ 
	\begin{array}{lc}
		x^{''} = f(t,x,x') \\
		x(a) = \alpha  \quad x(b)=\beta \\
	\end{array}
	\right.$$

    设区间$[a,b]$用点$a=t_0,t_1,t_2,...,t_{n+1}=b$分割,为简单起见,我们假设
    $$ t_i=a+ih \quad 0 \leq i \leq n+1 \quad h=\frac{b-a}{n+1}$$
    
    若$f$以非线性方式包含$y_i$,则这些方程是非线性的,一般来说求解很困难.因而我们总假定$f$关于$x$和$x'$是线性的,则它具有下列形式
              $$   f(t,x,x')=u(t)+v(t)x+w(t)x' $$
              
    以及下面的公式
    $$   x'(t)=(2h)^{-1}[x(t+h)-x(t-h)]-\frac{1}{6}h^2x^{'''}(\xi) $$
    $$   x^{''}(t)=h^{-2}[x(t+h)-2x(t)+x(t-h)]-\frac{1}{12}h^2x^{(4)}(\tau) $$
	
	那么我们可以将方程组写成下面的离散形式:
	
		$$	
	\left\{ 
	\begin{array}{lc}
		y_0 = \alpha \\
		(-1-\frac{1}{2}hw_i)y_{i-1}+(2+h^2v_i)y_i+(-1+\frac{1}{2}hw_i)y_{i+1} = -h^2u_i (1\leq i \leq n)\\
		y_{n+1} = \beta
	\end{array}
	\right.$$
	
	我们记$u_i=u(t_i),v_i=v(t_i).$
	
	下面我们引进缩写:
	$$ a_i=-1-\frac{1}{2}hw_{i+1}$$
	$$ d_i=2+h^2v_i$$
	$$c_i=-1+\frac{1}{2}hw_i$$
	$$b_i=-h^2u_i$$
	
	解线性方程组
	
		\begin{equation}
		\left[
		\begin{array}{cccccc}
			d_{1}& c_{1} &  &   \\
			a_{1}& d_{2} &c_{2}&   \\
			&a_{2} &d_{3}&c_{3}       \\
			&... & ...  &...&  \\
			&\quad&a_{n-2} &d_{n-1} &c_{n-1} \\
			&\quad&        &a_{n-1} &d_{n} 
			
		\end{array}
		\right ]
		\left[
		\begin{array}{cccc}
			y_{1}\\
			y_{2}\\
			y_{3}\\
			... \\
			y_{n-1}\\
			y_{n}
		\end{array}
		\right ]
		=
		\left[
		\begin{array}{cccc}
			b_{1}-a_0\alpha \\
			b_{2}\\
			b_{3}\\
			... \\
			b_{n-1}\\
			b_{n}-c_n\beta
		\end{array}
		\right ]
	\end{equation}

    解得$y_i$,作为近似的数值解.
	
	
	\section{Results}
	
	\begin{table}[h]
		\setlength\tabcolsep{25pt}
		\begin{tabular}{|l|l|l|l|l|}
			\hline
			n  &例1误差 & order  & 例2误差 & order  \\ \hline
			10  & 0.0047           & -----  & 2.4401e-04       & -----  \\ \hline
			20 & 0.0013          & 1.8650 & 6.7120e-05       & 1.8621 \\ \hline
			40 & 3.4022e-04       & 1.9317 & 1.7607e-05      & 1.9306 \\ \hline
			80 & 8.7174e-04       & 1.9645 & 4.5114e-06      & 1.9645 \\ \hline
		    160 & 2.2064e-04       & 1.9822 & 1.1419e-06      &1.9822 \\ \hline
		\end{tabular}
	\end{table}

  这里的误差是所有节点与解析解误差的最大值.
	

	
	
	\section{Discussion}
	
	我们可以看到有限差分法求解边值问题的误差随着$N$的增大,大致是$\mathcal{O}(h^2)$量级的.该方法的误差阶大致也是2阶的,并且比较稳定.这说明该方法的性质比较稳定,误差较小,性能优良.
	
	
	\section{Computer Code}
	
	\verbatiminput{main.m}
	\verbatiminput{differ.m}
	\verbatiminput{Chase_method.m}

\end{document}